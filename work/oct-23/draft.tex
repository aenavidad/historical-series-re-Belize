\documentclass{amsart}% see https://ctan.org/pkg/amsart or https://www.ams.org/arc/handbook/index.html - and may want to avoid taboos https://ctan.math.utah.edu/ctan/tex-archive/info/l2tabu/english/l2tabuen.pdf
%
%
%
%
% use only CTAN packages ? - amsthm, amsmath, amsfonts loaded  via amsart documentclass
% for tabes - csvsimple-l3 (for table import) seems only worthwhile for long tables - threeparttable (for table footers) seems to change spacing a little bit
\usepackage{array}% prereq of tabularx package so already loaded via that - apparently good practice to explicitly load nonetheless
\usepackage{amssymb}% for more ams symbols - may not need ?
% \usepackage[UKenglish]{babel}% for english variant
\usepackage{booktabs}% for table style
\usepackage[shortlabels]{enumitem}% for lists
\usepackage{lipsum}% for dummy text
\usepackage{tabularx}% for table spacing
\usepackage{tikz-cd}% tikz derivative - for diagrams
\usepackage[normalem]{ulem}% for strikethrough
% \usepackage{xcolor}% for colours
\usepackage{hyperref}% for hyperlinks - apparently must load sort-of last
\usepackage{amsrefs}% for citations - must load after hyperref
%
%
%
%
%
\newcommand{\code}[1]{\texttt{#1}}% for inline code - avoids formatting in document
\newcommand{\titleit}[1]{\textit{#1}}% for inline titles of books etc - avoids formatting in document
\newcommand{\mention}[1]{\textit{#1}}% for mentioned rather than used terms - avoids formatting in document
\newcommand{\tabnote}[1]{\footnotesize{#1}}% for macgyvered last-row-multicolumn table notes - avoids formatting in document
\newcommand{\simset}{\mathord\sim}% for \sim as set name rather than relation operator - fixes spacing
\newcommand{\powerset}{\raisebox{.15\baselineskip}{\Large\ensuremath{\wp}}}% for \powerset - avoids formatting in document
%
%
%
%
% the theorem styles below are general
\theoremstyle{plain}
\newtheorem{claim}{Claim}[section]
\theoremstyle{definition}
\theoremstyle{remark}
\newtheorem*{note}{Note}
%
%
%
%
% the theorem styles below are specially for a supplementary glossary
\swapnumbers% numbers in left instead of right
\theoremstyle{definition}
\newtheorem{term}{Term}[subsection]% definitions and elucidations - a sort of subsubsection for the glossary
\theoremstyle{remark}
\newtheorem*{term-type}{Type}
\newtheorem*{term-note}{Remark}
\newtheorem*{term-eg}{Examples}
\newtheorem*{term-ceg}{Counter-examples}
%
%
%
%
\newcolumntype{S}{>{\hsize=.6\hsize\linewidth=\hsize}X}% for tabularx package - may only use with one L columntype per ss 4.2-4.4 of package docs
\newcolumntype{L}{>{\hsize=1.4\hsize\linewidth=\hsize}X}% for tabularx package - may only use with one S columntype per ss 4.2-4.4 of package docs
%
%
%
%
%
%
%
%
\begin{document}
%
%
%
\title{Draft}
\author{AE Navidad}
\address{Harvard College, Cambridge, MA}% apparently research ie affiliation address
\curraddr{Belmopan, Belize}% if diff from \address
\email{navidad@college.harvard.edu}
\date{29 September 2023}
\thanks{\lipsum[1][1]}% appareantly rather for research grants info
\begin{abstract}
\lipsum[1][1-6]
\end{abstract}
\keywords{\lipsum[1][1]}
\maketitle
%
%
%
%
%
\section{Introduction}
\label{s:intro}
xx
%
%
%
%
\input draft-sec
%
%
%
%
%
\section{Context}
\label{s:cont}
xx
% sx of this section - edit this section to have the following sx per 1 oct 2023 13.15
% 1 - start w a short first-pass sketch of \Gamma\to W\loop, just enough to make 2 intelligible
% 2 - present sketch of \alpha_{-1} to \alpha_{1} transmission of information, and leave space for salient observations on this model
% 3 - may move to s:pres - *now* present a more thorough sketch of \Gamma\to W\loop, and in this case we need enough clarity to understand or see the series to be described in 4
% 4 - may move to s:pres - present and explain each series formally as a graph, and elucidate their subjects using the sketch from no 3

At this point, the following is proposed as a useful sketch of evidence used in historical reasoning.\footnote{Assume non-monotonic reasoning. Assume clean boundary between work and non-work things, for \mention{work} as \code{frbr:work} \sout{or \code{ioa:information content entity}} and \mention{thing} as \code{bfo:thing}. Do not assume clean break between evidence and product, for \mention{evidence} as \code{?\_0} and \mention{product} as \code{?\_1}. In the diagram we imagine first \(\alpha_0=\Gamma\to W\) giving us a set of work evidence \emph{only}, with later \(\alpha_1=W\to W\) resulting in product. Eg \(a_0\) might be a Mayan scribe's recording some claim \(c_0\) on some stela (say, `king so-and-so was crowned on 13.0.0.0.0'), while \(a_1\) might be an epigrapher's using \(c_0\) to arrive at some conclusion \(c_1\) (say, `king so-and-so was crowned on 13 August 516.')}% we want ?_0 evidence to be the output of some sort of minimal reasoning, and ?_1 product to require at least a bit more than minimal reasoning < 30 sep 2023 15.23
\[
\begin{tikzcd}
\Gamma \ar[r,"\alpha_0"] & W\ar[loop,"\alpha_1"] % \vdash for consequence ie follows from - \models for models
\end{tikzcd}
\]

for
\begin{align*}
\Gamma &= \text{set of non-work evidence,}\\
W &= \text{set of work evidence \emph{and} product.}
\end{align*}

Now, we might further sketch \(W\) as follows.\footnote{Assume product is always work output. Then \(\Gamma\) certainly has no work in it, and has no overlap with \(W\). Rather it contains events, states of affairs, slices of spacetime, and so on, eg non-work archaeological things (eg refuse middens, eg ceramic middens, eg ceramic shards, counter-eg ceramics% ceramic midden itself  is not a work, nor would ceramic shards themself, since midden nor shard's formation seem to count as work, but the ceramic itself would count as work it seems < 30 sep 2023 15.23
), or palaeolithic stuff (eg stalagmites).} Let \(\sim_0\) be the equivalence `is a manifestation just as' in \(W\), for \mention{manifestation} as \code{frbr:manifestation}. Then in \(\bigcup W\slash\simset_0=W_0\) we have all members of \(W\) which are manifestations, and in \(W_0^\prime\) all and only those which are not.\footnote{Eg unrecorded oral history still being realised. Note \(\sim_0\) might not actually be an equivalence, given how hazy \code{frbr:manifestation} is, and likewise for \(\sim_1,\ldots,\sim_3\).} Further, let \(\sim_1,\ldots,\sim_3\) be similar equivalence relations in \(W_0\) for \mention{published}, \mention{textual}, \mention{digital} as \code{?:1}, \code{?:2}, \code{?:3}.\footnote{For non-published we have eg manuscripts; for non-textual we might read \emph{mostly} non-textual eg maps, recordings, paintings; for non-digital we might understand manifestations with no digital item eg undigitised books, and for digital we would of course include born-digital.} Then in \(W_1\) we have all members of \(W_0\) which are published manifestations, and in \(W_1^\prime\) all and only those which are not, and so on. Lastly, let \(\sim_4,\sim_5\) be similar equivalence relations in \(W_1\) for \mention{historical}, \mention{official} as \code{?:4}, \code{?:5}. % maybe just allow all \sim_i for i>0 to not be equivalences ? seems like we only need \sim_0 as equivalence < 30 sep 2023 15.23
Then in \(W_4\) we have all members of \(W_1\) which are historical publications, and in \(W_4^\prime\) all and only those which are not, and so on.\footnote{Eg published books, articles on history or by historians in \(W_4\), and published papers, reports by Crown or parliament in \(W_5\).}

The following is a similar sketch.
\begin{verbatim}
level - set
0 - thing
1 - thing > work
1 - thing > non-work
2 - thing > work > manifestation
2 - thing > work > non-manifestation
3 - thing > work > manifestation > published
3 - thing > work > manifestation > non-published
3 - thing > work > manifestation > textual
3 - thing > work > manifestation > non-textual
3 - thing > work > manifestation > digital
3 - thing > work > manifestation > non-digital
4 - thing > work > manifestation > published > historical
4 - thing > work > manifestation > published > non-historical
4 - thing > work > manifestation > published > official
4 - thing > work > manifestation > published > non-official
\end{verbatim}

Now, given rough sketches of \(\Gamma\) and \(W\), we might move to sketching \(\alpha_0\) and \(\alpha_1\) as follows.\footnote{Where \(\alpha_{-1} = \Gamma\to\Gamma\), eg the witnessing of an event. %
%
% this process needs to be more fully worked out but seems fitting at 30 sep 2023 15.23
%
This sketch was made clearer by the \mention{continuum} or \mention{participation} model of scientific communication, which was brought to mind by Oliver Lugg.%
%
% ie back-and-forth transmission of science along intra-specialist -> inter-specialist -> pedagogical -> popular -> public audiences - seen 22 sep 2023 in belmopan at mark 1.36.12 of the `Mass Extinction Debates' YouTube video by Lugg
}
\begin{align*}
\alpha_{-1_0} &= \text{Mayan king witnesses his coronation} \\% \Gamma to \Gamma
\alpha_{0_0} &= \text{King informs scribe of coronation} \\% \Gamma to W given intentional utterance ie thing > work > non-manifestation
\alpha_{1_0} &= \text{Scribe records the coronation}\\% W to W but now work > non-manifestation to work > manifestation
\alpha_{1_1} &= \text{Epigraphist decodes coronation record}\\% eg gets `king was crowned on 13.0.0.0.0.0'
\alpha_{1_2} &= \text{Epigraphist translates decoded record}\\% eg gets `king was crowned on 13 Jan 605 Julian'
\alpha_{1_3} &= \text{Mayanist uses translated record to make claim}\\% eg gets `he was the n-th king of Caracol'
\alpha_{1_4} &= \text{Historian uses Mayanist claim to make claim}\\% eg gets `Caracol had at least n kings'
\alpha_{1_5} &= \text{Reviewer uses historical claim to make claim}\\% eg gets `the Mayan Lowlands had some p>>n kings'
\alpha_{1_6} &= \text{Professor uses reviewed claim to make claim}\\
\alpha_{1_7} &= \text{Journalist uses professorial claim to make claim}
\end{align*}

It seems each \(\alpha_i\) here introduces non-insignificant error into the stream. % eg loss or gaps or ommissions, distortions or inaccuracies or gluts or biases, so on - eg epigraphist gives 2 distinct translations, then Mayanist uses one for his claim, except the other may not have licensed his claim so well, etc
Further, it seems it would take much time and effort to trace the path back to the scribe's record, eg from the journalist's publication, in case one wanted to, say, fact check the journalistic claim or reasoning. It might prove useful, then, to bridge the \(\alpha_1\) path.% eg to specialists just for ease of reference ie not much gain for them - but to non-specialists and non-historians and broader public it would provide more than just ease of use ie very gainful for them - there should be more to observe / better point to make here it seems 30 sep 2023 16.16

One way this has historically been done within \(W_1\) is chronicles, ie chronological narratives of events, eg Peter Martyr's \titleit{Decades}. % ie in thing > work > manifestation > published - the 1516 Decades, for our area of interest
Of the same sort are calendars, catalogues, compendia, dictionaries, gazetteers, and the like, all of which are herein deemed \mention{series}.% this part seems to need a transition 30 sep 2023 17.12 - and a better sketch of what are and are not series 30 sep 2023 23.04

Table~\ref{t:series} lists those series which seem most useful to historical reasoning and dissemination or communication, and so desirable to have in version 1.0.0.
\input draft-tab% may want more local `Noted' examples for last col of this table eg Burdon's Archives, the like 30 sep 2023 22.44 - and may want separate \mention command for numbers to not italicise them ? 30 sep 2023 20.34
%
%
%
%
%
\section{Presentation}
\label{s:pres}
% sketch of all data series across all versions as one graph, possy in work/aug-22, as this will be used in supplement section for semver versioning 30 sep 2023 22.57
% we're going to fulfil tasks 3 and 4 from s:cont notes here ie -
% 3 - may move to s:pres - *now* present a more thorough sketch of \Gamma\to W\loop, and in this case we need enough clarity to understand or see the series to be described in 4
% 4 - may move to s:pres - present and explain each series formally as a graph, and elucidate their subjects using the sketch from no 3
% we might further want the series table in this section, maybe as text rather than a table.
We would now like a sketch of \mention{series}. A desirable sketch would be as simple as possible while nonetheless subsuming all and only those things we would like to count as series, both now and in the near future. So say a series is a pair \(S=(G,M)\) as follows.
\begin{align*}
S &= (G,M)\\
G &= (V,E)\\
V &= \{v_0,\ldots,v_k\}\text{ for labelled vertices }v_i\text{ and }0<k\\% functionally just finite strings ie datapoints
E &= \{e_0,\ldots,e_m\}\text{ for labelled edges }e_i\text{ and }0<m\\% functinally e_i = (l,v_i,v_j) ie the label = finite string, start point, end point
M &= \{m_0,\ldots,m_n\}\text{ for strings }m_i\text{ and }0\leq n% so allowed to be empty - this could be more complex eg strings and labelling functin to G and its subsets
\end{align*}
Here is one way to see \(S\). Say \(T\) is the set of finite, non-empty strings from a finite, non-empty alphabet. % so we have countable T
Eg we might have countable \(T=\{\langle 0\rangle,\langle 1\rangle,\langle 00\rangle,\langle 01\rangle,\ldots\}\) for the alphabet composed of symbols `0' and `1' only. Of course, a more fitting \(T\) would be one for Unicode characters. Then we might identify each vertex \(v_i\) with its label, and since the latter is just some \(t_j\) in some appropriate \(T\), we would have \(v_i=t_j\), and so \(V\subset T\). % proper subset
Similarly, we might imagine each edge \(e_i\) as a pair \(e_i=(t_j,(v_k,v_m))\) for label \(t_i\), edge start \(v_k\), and edge end \(v_m\), and so we would have \(E\subset T\times V^2\). Finally, as each \(m_i\) is just a string, we would have \(m_i=t_j\) and so \(M\subset T\).

We would now like to test the sketch. Consider Table~\ref{t:example}.
\begin{table}[h]% \centering not needed for amsart documentclass
\caption{\lipsum[1][1]}% must top table for amsart
\label{t:example}% should follow caption for amsart
\begin{tabularx}{\textwidth}{ccrrrX}
\toprule
id & year & start & value0 & value1 & note\\
\midrule
a000 & cal & 1 Jan 1630 & 13L3d0s & 1,000 & value0 via the KJB p 7, but disputed in KJB p 10\\
a001 & fis & 1 Mar 1630 & 13L3d0s & 970 & \lipsum[1][3]\\
\bottomrule
\addlinespace[\belowrulesep]% suggested by booktabs package
\multicolumn{6}{p{\textwidth}}{\tabnote{Sheet started on 1 oct 2021 in cambridge, ma. value0 via KJB pp 7-13, value1 via Wallace tab 3.}}
\end{tabularx}
\end{table}
Though there are a number of ways of translating such a table into a series, what matters to us is that there is some satisfactory and easy-enough translation. And we would like such an algorithm to exist, at least in concept, for any of our sources, including blocks of text.% more text goes here - we may want to revise this - we want to have a principled way of translating blocks of text and tables into graphs ie into series, as this is what we'll be focussing on from hereon out - and we need this sort of transformation in order to version everything else - perhaps we should focus on that for now 1 oct 2023 19.18 - furthermore, notet that *all* historical examples of series ought to count as series inc Burdon's archives, CSP, CDI, atlases, maps, even pictures and so one, so we need our concept of *series* to subsume them, and so we ought to have (i) general series subsuming any possible concept of series, (ii) formal series which is the shape our data will take ie graphs or graphs of graphs, in which case we need some princpled way of translating (i) to (ii), possibly via tables. 1 oct 2023 19.22 - though in the case of versioning, we ought to only include *structured* series so we have (i) series simpliciter subsuming all possible series, (ii) structured series including only *some* series eg human-readable tables (iii) graphical series ie a subset of structured series which meets further conditions eg having lots of metadata, being machine-readable, etc.

xx

We would now like a sketch of \mention{series}. We say a series is a graph \(G\) as follows.\footnote{That is, \(V=\{0,\ldots,m\}\), \(E=\{0,\ldots,n\}\), and \(d\colon E\to\powerset(V^2)\), for \(0<m,n\). By \mention{function} we mean total unary function. For the further avoidance of doubt -- let a total unary function \(f\colon A\to B\) be a partial unary function \(g\colon A\to B\) where, for any \(a\) in \(A\), there is at least one \(b\) in \(B\) such that \(f(a)=g(a)=b\); let a partial unary function \(g\colon A\to B\) be a binary relation \(\sim\) on \(A,B\) where, for any \(a\) in \(A\), if \(a\sim b_i\) and \(a\sim b_j\) then \(b_i=b_j\), which is to say, if \(g(a)=b_i\) and \(g(a)=b_j\), then \(b_i=b_j\); let a binary relation \(\sim\) on \(A,B\) be \(\simset\subseteq A\times B\).}
\begin{align*}
G &= (V, E, d)\\% ie a 3-tuple
V &= \{0,\ldots,m\}\text{ for } 0<m\\
E &= \{0,\ldots,n\}\text{ for } 0<m\\
d &\subseteq E\times\powerset(V\times V)\\
d(e) &=\begin{cases}
\{(v_{i_0},v_{j_0}),\ldots,(v_{i_n},v_{j_n})\} & \text{if}\\
\{\} & \text{else}
\end{cases}
\end{align*}
where for \(d\colon E\to\powerset(V\times V)\), \(d(e)=x\) if xx, else xx.
%
%
%
\subsection{Version 1.0.0}
\label{ss:ver1}
% find standards for in work/aug-22/ eg in work/aug-22/rules-20-aug-2022.md and work/aug-22/rules-20-aug-2022.tsv
% may want to modify standards as follows - merge the work/aug-22/ tripartite division of standards (work vs work-files vs 1.0.0 standards) into unified standards just for 1.0.0 - and now split these or specify these in increasing level of specificity eg going from the most generally-applicable standards which apply to the collection of series ie to entire dataset (eg have metadata, have doi, etc) down to standards applicable to each series as a whole (eg be human readable, be machine readable) down to standards applicable only to particular formats, datapoints, subseries, or parts of the whole (eg to datatypes, to metadata, to csv vs json vs online, to *forms of data manipulation* eg copying vs coding vs transforming) - 2 oct 2023 19.33
xx
%
%
%
\subsection{Version 0.n.0}
\label{ss:ver0}
xx
%
%
%
%
%
\section{Conclusion}
\label{s:conc}
xx
%
%
%
%
%
%
% \appendix - this is a command not environment - needed ? place before acknowledgements ? before refs ?
\section*{Supplements}
\label{s:supp}
xx
%
%
%
%
\subsection{Terms}
\label{ss:terms}
This is a non-alphabetical glossary of terms of interest to us. This is not a controlled vocabulary, though that would prove more useful. As such, terms are elucidated rather than strictly defined. We generally follow \href{https://www.ebi.ac.uk/ols4/ontologies/bfo}{\code{bfo}} and \href{https://www.ebi.ac.uk/ols4/ontologies/iao}{\code{iao}}, with deviations noted.
%
%
%
%
%
\begin{term}[Entity]
\label{term:entity}
An \href{http://purl.obolibrary.org/obo/BFO_0000001}{\code{bfo:entity}}.
\begin{term-note}
Primitive. Meant to subsume all instances in reality % bfo guide = `tracks instances in reality in a way that'
and all universals in reality and nothing else, for reality as the whole made of actual space and actual time and nothing else. Is exactly one of instance or universal. Not fully elucidated.\footnote{Not fully elucidated as \code{bfo} not specified for universals nor for instances not subject to study nor for instances not touching human endeavour. %
%
% in bfo particular = instance, universal = type, reality = space + time, category is a universal, exists is xx?
%
Partially elucidated as \code{bfo} claim entities are either particulars or universals, and claim no entity is both a particular and a universal.} See \href{https://doi.org/10.3233/978-1-60750-581-5-13}{doi:10.3233/978-1-60750-581-5-13} for xx.
\end{term-note}
\begin{note}
Certain examples of -- Julius Caesar, his body mass index, WWII, Verdi's \titleit{Requiem}. Certain non-examples -- mathematical things eg points eg numbers, propositions ie ideal meanings. Grey cases -- things with uncertain principles of identity?, spurious mereological sums?, universals of entities ie universals instantiated by universals?, things in possible worlds, xx.% eg those questionably `in reality' ie questionable existence - maybe some universals, maybe holes ? - we ought to restrict examples to our own historical domain
\end{note}
\begin{term-type}[Instance]
An entity which is a particular. Is not a universal. Is not relata of \code{bfo:is a}.
\end{term-type}
\begin{term-type}[Universal]
An entity which is a type. Is not a instance. Is instantiated by instances only. Is not instantiated by any universal. Is relata of \code{bfo:is a}. May be a subtype ie sub-universal.% ie no second order universals ie contains no lower universal
\end{term-type}
\end{term}
%
%
%
%
%
\begin{term}[Work]
\label{term:work}
An \href{http://purl.obolibrary.org/obo/IAO_0000030}{\code{iao:information content entity}} which \href{http://purl.obolibrary.org/obo/IAO_0000219}{\code{iao:denotes}} some instance.\footnote{Or, which was first \code{bfo:concretised} by some entity resulting from some \href{http://purl.obolibrary.org/obo/BFO_0000015}{\code{bfo:process}} where the latter \code{bfo:s-depends on} some \href{http://purl.obolibrary.org/obo/BFO_0000030}{\code{bfo:object}} which is a natural person. Or, which was created ie output by some \code{bfo:process} where the latter \ldots.}% or relate output itself directly to human person as opposed to via process ? - and may want to allow non-natural persons though might just reduce to natural ones - 2 oct 2023 06.42
\begin{term-note}[On \code{iao:ice}]
Derived via \href{http://purl.obolibrary.org/obo/BFO_0000031}{\code{bfo:gdc}} via \href{http://purl.obolibrary.org/obo/BFO_0000002}{\code{bfo:continuant}} from entity.\footnote{So imports as follows. An \code{iao:ice} \code{bfo:g-depends on} at least one \href{http://purl.obolibrary.org/obo/BFO_0000004}{\code{bfo:ic}}. Eg a poem (string of symbols) %
%
% so for ice it seems = `one` \neq `ONE' \neq `uno`
%
exists only so long as some physical copy of it exists somewhere, like as ink-on-paper or electrons-on-silicon or chemicals-in-neurons (ie the \code{iao:ice} may \mention{migrate} from copy to copy, in contrast to entities which \code{bfo:s-depend on} something, which cannot so migrate but are rather \mention{stuck} to a single copy). If an \code{iao:ice} \code{bfo:g-depends on} some \code{bfo:ic} then there is some \code{bfo:sdc} which \code{bfo:concretises} the \code{iao:ice} and which \code{bfo:s-depends on} the \code{bfo:ic}. %
% if x s-depends on y then x shares no parts with y - and x cannot exist sans y
% if x s-depends on y and y on z then x s-depends on z - ie transitive
% where no sdc is extended in space ie all sdc are immaterial - where an ic may be material or immaterial
% eg gdc = a string of symbols `one', sdc_0 = shape of black glyphs `o' and `n' and `e' in this screen here, sdc_1 = shape of black glyphs `o' and `n' and `e' in that screen there, such that sdc_i concretises gdc -- or if sdc_0 = pattern of inkmarks spread so-and-so on your manuscript, then sdc_0 concretises gdc, and further sdc_0 is a particular instance of gdc -- and if sdc_0 were taken to press and n copies typeset and printed such that sdc_1 is one print copy, ... , sdc_n is one print copy, then sdc_i would be a particular a particular instance of ic for 0<i
%
Eg say we had our poem and some physical copy of it somewhere (say ink on this-or-that parchment), then we would further have a depiction (arrangement of shapes and colours -- patterns of ink marks in this case) which exists only so long as that very physical copy of the poem exists. \sout{Eg \ldots if said physical copy (of the poem) were itself xeroxed (producing another copy of the poem), we would also have a copy of the visual image of the non-xeroxed copy of the poem.} %
%
% factcheck `and furthermore, ...' part re concretises - 2 oct 2023 15.35
% for concretises - if x g-depends on y, then there is some z which concretises x  and which s-depends on y - eg if ice poem g-deps on ic physical copy, then there is some sdc pattern which s-depends on the ic physical copy
%
If a \code{bfo:process} \code{bfo:has participant} \code{iao:ice}, then there is some \code{bfo:ic} which is not a \code{bfo:spatial region} such that \code{iao:ice} \code{bfo:g-depends on} it and \code{bfo:process} \code{bfo:s-depends on} it. Eg say some bard drafted our poem in this-or-that place and year, then we would have xx.} Meant to subsume xx. See \href{https://doi.org/10.3233/AO-210246}{doi:10.3233/AO-210246} for review, \href{https://hal.science/hal-03484145}{hal:03484145} for proposed extension to utterances, \href{https://ceur-ws.org/Vol-3155/short5.pdf}{ceur:v3155short5} for xx, \href{https://doi.org/10.3233/FAIA210370}{doi:10.3233/FAIA210370} for critique and proposed alternative. Not fully elucidated.\footnote{Not strictly defined as \code{iao} do not fully specify \href{http://purl.obolibrary.org/obo/IAO_0000136}{\code{iao:is about}} nor \href{http://purl.obolibrary.org/obo/IAO_0000015}{\code{iao:information carrier}}. % 
%
% which relation includes non-entity things as relata eg configurations
%
Partially defined as \code{iao} claim xx.}
\end{term-note}
\begin{term-note}[On \code{iao:denotes}]
Derived from \code{iao:is about}. Meant to subsume xx. Meant to exclude those \code{iao:ices} not relevant to us eg DNA sequences. Not fully elucidated.\footnote{Not fully elucidated as xx.}
\end{term-note}
\begin{note}
Certain examples of -- database entries input by some clerk, xx. Certain non-examples -- xx. Grey cases -- xx.% grey eg iao:ices which fail the iao:is_about relation eg fake news ? stories about myths ?
\end{note}
\end{term}
%
%
%
%
%
\begin{term}[Series]
\label{term:series}
A work which is published and scholarly and historical and xx.% but review w3c def https://www.w3.org/TR/vocab-dcat-3/#dataset-series - 4 oct 2023 04.07
\begin{term-note}
Derived. Meant to subsume xx.
\end{term-note}
\begin{note}
Certain examples of -- xx. Certain non-examples -- xx. Grey cases -- xx.
\end{note}
\begin{term-type}[Structured]
A series which is trivially transformable into an \(m\times n\) table for \(1<m\) columns and \(0<n\) rows. For instance, any series may be trivially transformed into a \(1\times 1\) table without much effort. On the other hand, any may be \emph{non}-trivially transformed into a \(m\times n\) table for \(1<m\) and \(0<n\). Take the first decade in Martyr's \titleit{Decades}. We might have this as a \(1\times 1\) table with a single datapoint containing the entire decade. Alternatively, since the decade is split into 10 books, we could have a \(2\times 10\) table with the first column naming the book, and the second giving the corresponding portion of the decade. And so on. In which case, we cannot trivially transform Martyr's first decade into the required table. On the other hand, we may trivially transform the \titleit{CSP} into one such, namely a \(2\times n\), with the first column indexing the paper (eg providing an id, creation date, creation place) and the second column describing the paper's contents. This last transformation is trivial in that this is already the format used in \titleit{CSP}.
\begin{term-type}[Digital]
A structured series which is machine-readable.% by which standard - eg structured data, has metadata, conforms to so-and-so standards - but in some transparent way ie excluding ISBN, MARC etc catalogue stuff as the `machine readable' part
\end{term-type}% digital
\end{term-type}% structured
\begin{term-type}[Empirical]
A series which is about xx ie not about a work nor its concretisations nor realisations eg chronicle, dictionary, gazetteer, compendium.
\end{term-type}
\end{term}
%
%
%
%
%
\subsection{Versioning}
\label{ss:versioning}
For \href{https://doi.org/10.5334/dsj-2021-012}{10.5334/dsj-2021-012}, we adopt the following versioning.
%
%
%
%
%
\subsection{Versioning2}
\label{ss:versioning2}
For \href{https://semver.org}{\code{semver 2.0.0}}, as interpreted for data by xx, we have the following versioning.% need semver interpretation for data esp non-machine readable data before we can use sep-23/draft0-s-7-sources.tsv to fill out this versioning section 30 sep 2023 22.21
\footnote{There is apparently no consensus yet on best practice for versioning data per \href{http://doi.org/10.5334/dsj-2021-012}{doi:10.5334/dsj-2021-012} and \href{https://www.w3.org/TR/dwbp/}{w3c:dwbp}.}% info in klump intro para 5 and w3c s 8.6 para 1, klump ref from work/dec-22/free-text-nov-25.md s 'log' no 2 - w3c dcat defers to dwbp per s 11 https://www.w3.org/TR/vocab-dcat-3/
\begin{enumerate}
\item semantic version 1.0.0, previously version 1.5, that dataset described in section \ref{ss:ver1}, immediately preceded (possibly with alpha's and rc's) by 0.r.s for \(\text{p}<\text{r}\) and \(0\leq\text{s}\),
\item semantic versions 0.p.q to 0.r.s,
\item semantic version 0.n.0 for \(0<\text{n}\), % possy 99 < n << 0
previously version 1.4, that dataset described in section \ref{ss:ver0}, immediately followed (possibly with alpha's and rc's) by 0.p.q for \(\text{n}<\text{p}\) and \(0\leq\text{q}\),
\item semantic versions \ldots,
\item semantic version 0.0.0, that dataset xx.% eg first tsv file ? first tsv file group ? gotta find semver interpretation for data
\end{enumerate}
However, it might prove more practical to carve out all data since the earliest \code{tsv} file on record, so that we might retroactively form the calendar, catalogue, etc series, and assign a version to each series \emph{itself}, rather than to all of them as a whole. This done, it might be easier to come up with a version for the dataset as a whole.% though it seems we would need one graph for each such series, rather than one giant graph as envision in work/aug-22 or similar files ? - nevertheless, semver for each series across time since earliest tsv, rather than for *all* series jointly, seems acceptable - also, earliest record might count as a single vertex eg for chronicle, but not sure how appropriate that is . 1 oct 2023 00.55 - further it might be legal to skip version numbers eg x.y.z -> x.y+1.z might not be required st x.y.z -> x.y+2.z might be legal, and this would be more useful in this case than non-gappy versioning - but we still need the formal sketch of data as a graph, just this time for series rather than entire dataset . 1 oct 2023 01.04
%
%
%
%
%
\section*{Acknowledgements}
\label{s:acknow}
xx
%
%
%
%
\input draft-refs % dummy refs
%
%
%
%
%
\end{document}