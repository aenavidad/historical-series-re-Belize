\documentclass{amsart}
%%%
%%%
%%%
%%%
%%%
%%%
%%%
\usepackage{hyperref,tikz-cd}
\usepackage[UKenglish]{babel}
\usepackage[shortlabels]{enumitem}
%%%
%%%
%%%
%%%
%%%
%%%
%%%
%%%
%%%
%%%
%%%
%%%
%%%
\linespread{1.3}
%%%
%%%
%%%
%%%
%%%
%%%
%%%
%%%
%%%
%%%
\title{1638 / 1.0}
\author{Navidad}
\date{\today}
%%%
%%%
%%%
%%%
%%%
%%%
%%%
%%%
%%%
%%%
\begin{document}
\maketitle
%%%
%%%
%%%
%%%
%%%
%%%
%%%
%%%
%%%
%%%
%%%
%%%
%%%
%%%
%%%
%%%
%%%
\section{Pre}
%%%
%%%
%%%
%%%
\subsection{Space}
Use the usual geometric model spacetime. Have events modeled as shapes and agents as line segments. Geographic spaces as shapes too. Divide objects of enquiry or discussion like.
	\[
 	\begin{tikzcd}[column sep=small]
				& S\ar[dl] \ar[dr]	& 	&	&\\
	P^\prime	&					& P\ar[dl]\ar[dr]&&\\
				&			A^\prime& 	&A\ar[dl]\ar[dr]&\\
				&					&T^\prime&	&T
	\end{tikzcd}
	\]
where \(S\) is all of spacetime, \(P\) is a clean cut of spacetime restricted by geographic and temporal boundaries, \(A\) is a fixed restricted list of line segments ie agents, \(T\) is a fixed class of events.
%%%
%%%
%%%
%%%
%%%
%%%
%%%
%%%
%%%
%%%
\subsection{P limits}
%%%
%%%
For \(P\subset E\) we want to take a clean cut of \(E\). To make this, we want to know where to cut ie start and end for the temporal dimension, and \(x, y, z\) coordinates for the extended dimensions.
\subsubsection{Temporal limits} We want to start at the time when Bacalar was estblished ie March \(1543\) so say from and including \(1\) January \(1543\) Gregorian ie NS. We want to stop some short while after the Capture of Tris so say to and including \(31\) December \(1717\) Gregorian ie NS. Giving use \(175\) years.
\subsubsection{Extended limits} We want to include \(z\) ie height above and below the area to be demarcated to some small extent so say \(1\) mile of airspace above some arbitrary sealevel and \(1\) mile of subspace below the same arbitrary sealevel eg as measured form Earth's centre core assuming prefectly spherical globe. Now we give vertices of an \(n\)-gon on said sphere by giving breadth and width points ie \(x,y\) coordinates in the form of GPS coordinates. Eg going clockwise -
	\begin{enumerate}
	\item Start at some point currently in Nicaragua ie the southernmost point of the greatest extent the Mosquito Shore ever had eg mouth of Rio San Juan,
	\item Run along the midpoint of the Caribbean and Pacific coasts of the Central American landmass or isthmus until first reaching or hitting a point in Peten, Guatemala,
	\item Take a beeline to the first or southermost point on Rio Chixoy which forms part of the Mexico-Guatemala border,
	\item Follow this border until reaching the Tabasco-Chiapas border,
	\item Follow that border until reaching the coast,
	\item From this point make a northern beeline some good number of nautical miles into the Gulf of Mexico,
	\item Trace the coast from this point, around the Yucatan peninsula, all the way to Cape Camaron Light in present-day Honduras, being sure to include offshore cayes etc,
	\item From this point of Camaron make a beeline to some point a good number of natical miles north of the Cayman Islands,
	\item And make another beeline to the easternmost point north a good number of nautical miles north of the easternmost point of Jamaica, and trade that coast to its westernmost point,
	\item See map for more as got tired.
	\end{enumerate}
%%%
%%%
%%%
%%%
%%%
%%%
%%%
%%%
%%%
%%%
%%%
\subsection{A limits}
%%%
%%%
%%%
%%%
%%%
%%%
%%%
%%%
%%%
%%%
%%%
%%%
%%%
%%%
%%%
%%%
\subsection{T limits}
%%%
%%%
%%%
%%%
%%%
%%%
%%%
%%%
%%%
%%%
%%%
%%%
%%%
%%%
%%%
%%%
%%%
\subsection{T contents}
%%%
%%%
%%%
%%%
%%%
%%%
%%%
%%%
%%%
%%%
%%%
%%%
%%%
%%%
%%%
%%%
%%%
%%%
%%%
%%%
%%%
%%%
%%%
%%%
%%%
%%%
%%%
%%%
%%%
%%%
%%%
%%%
%%%
%%%
%%%
%%%
%%%
%%%
%%%
%%%
%%%
%%%
%%%
%%%
%%%
%%%
%%%
%%%
%%%
%%%
%%%
%%%
%%%
%%%
%%%
\end{document}
